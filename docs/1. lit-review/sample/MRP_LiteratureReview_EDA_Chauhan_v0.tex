%%%%%%%%%%%%%%%%%%%%%%%%%%%%%%%%%%%%%%%%%%%%%%%%%%%%%%%%%%%%%%%%%%%%%%%%%%%
% This is a sample template for the graduate thesis report for the Master of Science in Data Science program at Thompson Rivers University, Canada, created by Bhavithry Sen Puliparambil in April 2022. Please refer to the graduate research department and MSc Data Science Online resources for updates in the TRU Thesis Style Guide. https://www.tru.ca/research/graduate-studies/resources-for-graduate-students/thesis-style-guide.html
%%%%%%%%%%%%%%%%%%%%%%%%%%%%%%%%%%%%%%%%%%%%%%%%%%%%%%%%%%%%%%%%%%%%%%%%%%%%

\documentclass[a4paper]{article}

%% Language and font encodings
\usepackage[utf8]{inputenc}
\usepackage{float}
\usepackage{booktabs}
\usepackage{tabu}
\usepackage{url}

%% Sets page size and margins
\usepackage[a4paper,top=2cm,bottom=2cm,left=2cm,right=2cm,marginparwidth=1.75cm]{geometry}

%% Useful packages
\usepackage{amsmath}
\usepackage{graphicx}
%\usepackage{apacite}
\usepackage[colorinlistoftodos]{todonotes}
\usepackage[colorlinks=true, allcolors=blue]{hyperref}

\documentclass[12pt,a4 paper,openany,oneside, title page]{book}
\usepackage[utf8]{inputenc}

\usepackage{titlesec}
\usepackage{fancyhdr} 

% Spacing
\usepackage[english]{babel}
  \setlength{\parindent}{20pt}
  \setlength{\parskip}{1em}
  %\linespread{1.3}
  \renewcommand{\baselinestretch}{1.5}
% Bibliography
\bibliographystyle{plain}

% Figures
\usepackage{graphicx}
\graphicspath{ {figures/} }
\usepackage{array}
\usepackage{float}
\usepackage[colorlinks = true,
            linkcolor = blue,
            urlcolor  = blue,
            citecolor = black,
            anchorcolor = blue]{hyperref}
\usepackage{wrapfig}
  \setlength{\pdfpagewidth}{9in} \setlength{\pdfpageheight}{12in}

\usepackage{amsmath}
\DeclareMathOperator*{\argmin}{argmin}

\usepackage{ragged2e}
\usepackage{hyperref}
\usepackage{epstopdf}
\usepackage[nolist, nohyperlinks]{acronym}


\usepackage{algorithm}





\begin{document}


\begin{titlepage}
\centering
{\Large\bfseries LLM-Augmented Knowledge-Graph-Based Recommendation System}

% \vspace{1.5cm}

% {\Large [Title]}

% \vspace{1cm}

{By}

\vspace{0.5cm}

{\large Kartikey Chauhan

501259284 }

\vspace{1cm}

{\large Literature Review \& Exploratory Data Analysis 
}

\vspace{1cm}

{\large Master of Science 
in the Program of 
Data Science and Analytics
}

\vspace{0.5cm}
{\large  Toronto, Ontario, Canada, 2024}

% \vspace{0.5cm}
% {[Month, Year]}

% \vspace{0.5cm}
% {SUPERVISOR} 

% {[Name]}



\vfill

{\itshape © Kartikey Chauhan, 2024}
\end{titlepage}


\pagenumbering{roman}
\setcounter{page}{2}

\thispagestyle{fancy}
\fancyhf{}
\renewcommand{\headrulewidth}{0pt}
\cfoot{\thepage}
\centering{\bfseries ABSTRACT}


\begin{justifying}
[This page is required for all graduate theses. An abstract is a short paragraph explaining the
major points and conclusions of your thesis. For master's theses, the abstract can be no more
than 150 words long.
The word, “ABSTRACT”, is centered at the top of the page, all caps.
The abstract presents an abbreviated account of the thesis including purpose, methodology, results and conclusions. It needs to be dense with information but also well-written, wellorganized and self-contained, without references, abbreviations, acronyms, jargon, figures or tables. Abstracts may not exceed one page or 150 words.
]

\noindent{Key Words: [Key word1; Key word2; Key word3; Key word4; Key word5].}
\end{justifying}



\newpage
\thispagestyle{fancy}
\renewcommand{\headrulewidth}{0pt}
\cfoot{\thepage}
\centering
{\bfseries ACKNOWLEDGEMENTS}

\justifying
\noindent{Optional page, not to exceed one page in length.
The word, “ACKNOWLEDGEMENTS”, is centered at the top of the page, all caps. Roman numeral page number at the bottom of the page, centred.}

\tableofcontents
\listoffigures
\listoftables
\newpage

% Use Arabic numbers starting with the first page of chapter 1, and continue consecutively to the end of the thesis, including appendices. Page numbers are placed in the lower right of the page, one inch from the lower and right-hand margin, using the same typeface as the text.
\pagestyle{fancy}
\fancyhf{}
\renewcommand{\headrulewidth}{0pt}
\fancyfoot[R]{\thepage}
\renewcommand{\footrulewidth}{0pt}
\pagenumbering{arabic}



\chapter{Introduction}

This document covers the Introduction, Literature Review and Exploratory Data Analysis for
the first deliverable of my Major Research Project (MRP). It begins with a brief background on
the topic and datasets, defines the problem, and states the research question. This is followed
by a literature review and a detailed exploratory analysis of the dataset.

\section{Background}
Recommender systems have been widely applied to address the issue of information overload in various internet services, exhibiting promising performance in scenarios such as e-commerce platforms and media recommendations. 
In the general domain, the traditional knowledge recommendation method is  \textit{collaborative filtering (CF)}, which usually suffers from the cold start problem and sparsity of user-item interactions. 
Knowledge-based recommendation models effectively alleviate the data sparsity issue leveraging the side information in the knowledge graph, and have achieved state of the art performance\cite{guo2020survey} (e.g. KGAT\cite{wang2019kgat} , LightGCN\cite{he2020lightgcn}).
However, KGs are difficult to construct and evolve by nature, and existing methods often lack considering textual information. On the other hand, LLMs are black-box models, which often fall short of capturing and accessing factual knowledge.
Therefore, it is complementary to unify LLMs and KGs together and simultaneously leverage their advantages (See \hyperref[fig:llm_vs_kg]{Figure 1}).
This project aims to explore LLM-augmented KGs, that leverage Large Language models (LLM) for different KG tasks such as embedding, completion, construction and also incorporate textual information which could be a way to help overcome these challenges and lead to better recommendation systems.

\subsection{Research Objectives}
\item Can LLM's be used to enhance the construction / quality / volume of information in the knowledge graphs? Is it possible to effectively constrain LLM output to be of a specific systematic knowledge extraction format?
\item Do these improved knowledge graphs lead to better recommendation systems? 
\item Can we combine current SOTA methods with the use of LLMs in extracting latent relationships, Knowledge Graph (KG) embedding, KG completion, and KG construction for recommendation in an efficient, explicit, and end-to-end manner.


\section{Figures and Tables}
Every table and figure/photograph/illustration must have a caption, normally indicating what the item is intended to illustrate or represent. For instance, Fig.~\ref{fig1} shows the Thompson Rivers University Logo downloaded from the univsersity's website. 
Tables are numbered consecutively throughout the thesis.
Figures/photos/illustrations are also numbered consecutively throughout the thesis.

\begin{figure}
\includegraphics[width=\textwidth]{TRU_Logo.png}
\caption{This is the logo of Thompson Rivers University. [Your description of the figure is to be included here.]} \label{fig1}
\end{figure}

\chapter{Literature Review}


• Literature Review (15-25 pages)

In this section, I provide an overview of the papers referenced for this project.
There have been several efforts to construct KGs to facilitate the discovery of relevant information within specific fields.


\chapter{Methodology}
• Theory Development (Explanation of the new model/algorithm) (10-20 pages)


Lorem ipsum dolor sit amet, consectetur adipiscing elit, sed do eiusmod tempor incididunt ut labore et dolore magna aliqua. Ut enim ad minim veniam, quis nostrud exercitation ullamco laboris nisi ut aliquip ex ea commodo consequat. Duis aute irure dolor in reprehenderit in voluptate velit esse cillum dolore eu fugiat nulla pariatur. Excepteur sint occaecat cupidatat non proident, sunt in culpa qui officia deserunt mollit anim id est laborum.Lorem ipsum dolor sit amet, consectetur adipiscing elit, sed do eiusmod tempor incididunt ut labore et dolore magna aliqua. Ut enim ad minim veniam, quis nostrud exercitation ullamco laboris nisi ut aliquip ex ea commodo consequat. Duis aute irure dolor in reprehenderit in voluptate velit esse cillum dolore eu fugiat nulla pariatur. Excepteur sint occaecat cupidatat non proident, sunt in culpa qui officia deserunt mollit anim id est laborum.


\chapter{Discussion}
• Discussion/application (15-25 pages)


Lorem ipsum dolor sit amet, consectetur adipiscing elit, sed do eiusmod tempor incididunt ut labore et dolore magna aliqua. Ut enim ad minim veniam, quis nostrud exercitation ullamco laboris nisi ut aliquip ex ea commodo consequat. Duis aute irure dolor in reprehenderit in voluptate velit esse cillum dolore eu fugiat nulla pariatur. Excepteur sint occaecat cupidatat non proident, sunt in culpa qui officia deserunt mollit anim id est laborum.Lorem ipsum dolor sit amet, consectetur adipiscing elit, sed do eiusmod tempor incididunt ut labore et dolore magna aliqua. Ut enim ad minim veniam, quis nostrud exercitation ullamco laboris nisi ut aliquip ex ea commodo consequat. Duis aute irure dolor in reprehenderit in voluptate velit esse cillum dolore eu fugiat nulla pariatur. Excepteur sint occaecat cupidatat non proident, sunt in culpa qui officia deserunt mollit anim id est laborum.





\chapter{Conclusion}
• Conclusion (5-10 pages)


Lorem ipsum dolor sit amet, consectetur adipiscing elit, sed do eiusmod tempor incididunt ut labore et dolore magna aliqua. Ut enim ad minim veniam, quis nostrud exercitation ullamco laboris nisi ut aliquip ex ea commodo consequat. Duis aute irure dolor in reprehenderit in voluptate velit esse cillum dolore eu fugiat nulla pariatur. Excepteur sint occaecat cupidatat non proident, sunt in culpa qui officia deserunt mollit anim id est laborum.Lorem ipsum dolor sit amet, consectetur adipiscing elit, sed do eiusmod tempor incididunt ut labore et dolore magna aliqua. Ut enim ad minim veniam, quis nostrud exercitation ullamco laboris nisi ut aliquip ex ea commodo consequat. Duis aute irure dolor in reprehenderit in voluptate velit esse cillum dolore eu fugiat nulla pariatur. Excepteur sint occaecat cupidatat non proident, sunt in culpa qui officia deserunt mollit anim id est laborum.






\clearpage
\nocite{*}
\bibliography{refs}


\appendix
\chapter{Appendix Heading}

Lorem ipsum dolor sit amet, consectetur adipiscing elit, sed do eiusmod tempor incididunt ut labore et dolore magna aliqua. Ut enim ad minim veniam, quis nostrud exercitation ullamco laboris nisi ut aliquip ex ea commodo consequat. Duis aute irure dolor in reprehenderit in voluptate velit esse cillum dolore eu fugiat nulla pariatur. Excepteur sint occaecat cupidatat non proident, sunt in culpa qui officia deserunt mollit anim id est laborum.

\end{document}