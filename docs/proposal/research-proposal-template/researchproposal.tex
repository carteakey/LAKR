\documentclass[
    ngerman,american
    ]{scrartcl}

    % ##########################################
    % # Choose the language for the document by editing below line
    % # de = German
    % # en = English
    \newcommand{\lang}{en}
    % ##########################################

    \usepackage{babel}
    \usepackage[utf8]{inputenc} 
    \usepackage{csquotes}
    \usepackage{enumitem}
    \usepackage{ifthen}
    \usepackage{lipsum}
    
    \newcommand{\paperSubTitle}[1]
{
    \ifthenelse{\equal{#1}{en}}{Outline and Topic Proposal}{}
    \ifthenelse{\equal{#1}{de}}{Outline und Themenvorschlag}{}
}

\newcommand{\sectionQuestions}[1]
{
    \ifthenelse{\equal{#1}{en}}{\section{Scope of Work - 4 Questions}}{}
    \ifthenelse{\equal{#1}{de}}{\section{Ziel der Arbeit - 4 Fragen}}{}
}

\newcommand{\sectionQuestionsDescription}[1]
{
    \ifthenelse{\equal{#1}{en}}{In this section the essence of the proposed work is described by answering four key questions. }{}
    \ifthenelse{\equal{#1}{de}}{Im Folgenden wird der Kern der Arbeit beschrieben indem vier Kernfragen beantwortet werden.}{}
}

\newcommand{\sectionInitialTOC}[1]
{
    \ifthenelse{\equal{#1}{en}}{\section{Preliminary Table of Contents}}{}
    \ifthenelse{\equal{#1}{de}}{\section{Vorläufige Gliederung}}{}
}

\newcommand{\sectionInitialTOCDescription}[1]
{
    \ifthenelse{\equal{#1}{en}}{In this section the table of contents for the proposed work is described.}{}
    \ifthenelse{\equal{#1}{de}}{Im Folgenden wird ein Inhaltverzeichnis für die vorgeschlagene Arbeit vorgestellt.}{}
}

\newcommand{\sectionSource}[1]
{
    \ifthenelse{\equal{#1}{en}}{\section{Relevant Related Work}}{}
    \ifthenelse{\equal{#1}{de}}{\section{Relevante verwandte Arbeiten}}{}
}


\newcommand{\sectionSourceDescription}[1]
{
    \ifthenelse{\equal{#1}{en}}{In this section, identified related work is described.}{}
    \ifthenelse{\equal{#1}{de}}{Diese Section stellt verwandte Arbeiten dar und erklärt kurz deren Bedeutung für die vorgeschlagene Arbeit.}{}
}

\newcommand{\questionOne}[1]
{
    \ifthenelse{\equal{#1}{en}}{What is the problem you want to address in your work?}{}
    \ifthenelse{\equal{#1}{de}}{Was ist das Problem, welches Sie in Ihrer Arbeit bearbeiten wollen?}{}
}

\newcommand{\questionTwo}[1]
{
    \ifthenelse{\equal{#1}{en}}{Why is it a problem?}{}
    \ifthenelse{\equal{#1}{de}}{Warum ist es ein Problem?}{}
}

\newcommand{\questionThree}[1]
{
    \ifthenelse{\equal{#1}{en}}{What is the solution you developed in your work?}{}
    \ifthenelse{\equal{#1}{de}}{Was ist die Lösung die sie entwickelt haben?}{}
}

\newcommand{\questionFour}[1]
{
    \ifthenelse{\equal{#1}{en}}{Why is it a solution?}{}
    \ifthenelse{\equal{#1}{de}}{Warum ist es eine Lösung?}{}
}


    \ifthenelse{\equal{en}{\lang}}
    {
        \selectlanguage{american} 
    }{
        \ifthenelse{\equal{de}{\lang}}
        {
            \selectlanguage{ngerman}
        }
        {\selectlanguage{american}}        
    }

    \usepackage[
        bibencoding=utf8, 
        style=alphabetic
    ]{biblatex}

    \bibliography{bibliography}
    
    
    \usepackage{amsmath}
    \title{
        % ##########################################
        % # Insert the title of your paper/thesis here
        % ###### 
        % Coming up with a good title is hard.
        % It should:
        %  1. capture the contents of the your work
        %  2. not be to broad or generic
        %  3. stick to the truth and don't not oversell
        %  4. use established terms and wordings
        %  5. make people curious about your work
        %  6. use current buzzwords if possible (but do it right)
        %  7. not use too many buzzwords :-)
        LLM-Augmented Knowledge Graph-Based Recommendation Systems
        % ##########################################
        \\  \Large{\paperSubTitle{\lang}}} % don't touch this line

    \author{
        % ##########################################
        % # Your name goes here
        % ######
        % well, that should be obvious, right? 
        Kartikey Chauhan
        % ##########################################
        }
    
    \begin{document}
      \maketitle
        \begin{abstract}

            \ifthenelse{\equal{en}{\lang}}{testen}{}
            % ##########################################
            % # Include your Abstract here 
            % ######
            % I would strongly suggest to start working on the abstract only 
            % after you have answered the 4 questions in Section 1, as this will
            % make it much easier for you to come up with an abstract that
            % is to the point, short, and still summarizing all the most crucial 
            % results of your work.
            %
            % The abstract should include the following points:
            %  - a short but to the point introduction of the problem area
            %  - what is the topic/problem, tackled in your work? 
            %  - why is the topic/problem of your work relevant? Why should the 
            %    reader care about it?
            %  - what are the results/answers of your work?
            %  - how did you gain your results and what is their quality?
            %                %  
            % It should NOT be:
            %  - too long/verbose
            %  - too short
            \lipsum[1-2]
            % ##########################################

        \end{abstract}
        
        
        \sectionQuestions{\lang}
        \sectionQuestionsDescription{\lang}
        
        \begin{description}[style=unboxed]
            \item [\questionOne{\lang}] 
                % ##########################################
                % # Question 1: What is the problem you want to address in your work? / 
                %               Was ist das Problem, welches Sie in Ihrer Arbeit bearbeiten wollen?
                % ######
                % The goal of this question is to clearly state what your work is about. 
                % What is the problem it is supposed to solve?
                %
                % Answering this question is particular important during the early phases 
                % of your work, in order to gain further insight and understanding about what 
                % your work is going to cover and address.  
                %
                % Answer this question very briefly by stating the problem or research 
                % question that you want to address/solve in your work.
                % 
                % Your answer should: 
                %  - only be 1 sentence (2 sentences max)
                %  - not cover a statement why the topic is relevant 
                %    for the industry (this is address by the next question)
                %  - properly use common terms and buzzwords of IT today (similar to the 
                %    rules for the abstract)
                %
                % Please acknowledge: the answer to this question should not cover why the 
                % problem is important or relevant to anyone (e.g. industry). This will 
                % be addressed with the next question.
                insert answer here
                % ##########################################

            \item [\questionTwo{\lang}]
                % ##########################################
                % # Question 2: Why is it a problem? / Warum ist es ein Problem?
                % ######
                % The goal of this question is to describe why your work is relevant. 
                % Why should the reader care? Why is this the problem (of question 1)
                % worth investigating?
                %
                % Answering this question is particular important during the early phases 
                % of your work, in order to gain further insight and understanding of the 
                % problem domain you are addressing. Further, it is a good checkpoint to 
                % ensure that you are addressing issues that are not just theoretical but 
                % have real-world applications. 
                %
                % Your answer should: 
                %  - be 3 - 5 sentences 
                %  - give a broader overview of the domain/area where your problem occurs
                %  -- who has this problem?
                %  -- what is the impact of it?
                %  -- which conditions need to be fulfilled for the problem to occur?
                %  -- etc.
                %  - describe the benefit of having the problem resolved
                insert answer here
                % ##########################################

            \item [\questionThree{\lang}]
                % ##########################################
                % # Question 3: What is the solution you developed in your work? / 
                %               Was ist die Lösung die sie entwickelt haben?
                % ######
                % The goal of this question is to describe the results of your work 
                % ans/or solution to the problem of your work.
                % 
                % It is hard/impossible to answer this question in the early phases of 
                % your work, as usually you do not have results, yet. However, you can 
                % already state first ideas that you may have in order to discuss them 
                % with your supervisor. 
                %
                % Your answer should:
                %  - clearly state all results of your work, that are relevant to your 
                %    research problem. 
                %  - not oversell your results, stick you what you actually have 
                %    accomplished"
                %  - give credit where credit is due. If you created your results based 
                %    on the work of others, give them the credit.
                %  - if none of your ideas did not produce any usable solution, state 
                %    so - these attempts are also results! By documenting them, it may 
                %    prevent others from trying them as well. 
                insert answer here
                % ##########################################

            \item [\questionFour{\lang}]
                % ##########################################
                % # Question 4: Why is it a solution? / Warum ist es eine Lösung?
                % ######
                % The goal of this question is to describe who you developed your results 
                % and what the quality of them are.
                % 
                % Your answer should:
                %  - short and to the point
                %  - clearly state how you developed your results
                %  -- what is the chain of reasoning that led to your results/solution
                %  -- what statistics, literature, studies, or other literature did you 
                %     base your assumptions on?
                %  - clearly state the quality and applicability of your results
                %  -- reflect your work objectively - there is no perfection in this 
                %     world, so your work is not perfect as well. Be aware of that!
                %  -- how did you ensure that your results are accurate? did you:
                %  --- perform experiments? 
                %  --- apply any logical deductions?
                %  --- mathematical proofs? 
                %  --- implement a "proof of concept" implementation and evaluate? 
                %  --- etc.
                %  - clearly state the shortcomings of your work
                %  -- be hones and objective about your own work. 
                %  -- In which cases/scenarios are your results applicable?
                include answer here
                % ##########################################
        \end{description}
        
        \sectionInitialTOC{\lang}
        \sectionInitialTOCDescription{\lang}
        
        % ##########################################
        % # Proposed table of contents
        % ######
        % The goal of this section is to propose a table of contents. Please keep in
        % mind: a well created table of contents is very powerful in provide a good 
        % overview of the overall chain of reasoning of your work. This makes it 
        % extremely valuable.
        % 
        % Please include:
        %  - names of sections and subsections (please don't go deeper than that unless 
        %    your supervisor asks you for it)
        %  - a brief description of the proposed content of each section and subsection
        %    (1-3 sentences)
        % 
        \begin{enumerate}
            \item \textbf{Section 1 Name} insert brief description
                    \begin{enumerate}
                        \item \textbf{Subsection 1 Name} insert brief description
                        \item \textbf{Subsection 2 Name} insert brief description
                    \end{enumerate}
            \item \textbf{Section 2 Name} insert brief description
        \end{enumerate}
        % ##########################################
    
      
        \sectionSource{\lang}
        \sectionSourceDescription{\lang}

        % ##########################################
        % # Overview of identified relevant work
        % ######
        % The goal of this section is to provide an overview of the relevant and significant 
        % related work identified so far. Make sure that your cited sources are of appropriate
        % quality!
        % 
        % Please include:
        %  - a citation of the source using Latex facilities (incl. a generated list of 
        %    references)
        %  - a brief descriptions of the source and a statement why this is relevant for 
        %    your work (1-2 sentences)
        % 
        \begin{description}
        \item[\cite{gruba_how_2017}] insert brief description
        \item[\cite{zobel_writing_2015}] insert brief description
        \end{description}
        % ##########################################
        
        
      \printbibliography
    \end{document}